\lstset{style=fsharpstyle}

\section{Используемые технологии} 
\label{sec:practice:technology_used}

\subsection{Язык Python}

Python — высокоуровневый язык программирования общего назначения с динамической типизацией и автоматическим управлением памятью, ориентированный на повышение производительности разработчика, читаемости кода и его качества, а также на обеспечение переносимости написанных на нём программ. Язык является полностью объектно-ориентированным --- всё является объектами. Синтаксис ядра языка минималистичен, за счёт чего на практике редко возникает необходимость обращаться к документации. Сам же язык известен как интерпретируемый и используется в том числе для написания скриптов. Недостатками языка являются зачастую более низкая скорость работы и более высокое потребление памяти написанных на нём программ по сравнению с аналогичным кодом, написанным на компилируемых языках. 

Python — многоцелевой язык, и его можно использовать для создания практически чего угодно. Компании по всему миру используют Python для создания искусственного интеллекта и решения задач машинного обучения, разработки веб-сайтов, различных вычислений, игр а также многих других целей.

Что касается искусственного интеллекта, Python стоит выше других языков программирования и считается лучшим для приложений на базе ИИ. По сути, Python решает любые задачи ИИ: машинное обучение, анализ и визуализация данных, обработка естественного языка и компьютерное зрение.

Проекты ИИ отличаются от традиционных программных проектов в стеке технологий, требуемых навыках и необходимости глубоких исследований. Для реализации подобных приложений, нужно использовать язык программирования, который является стабильным, гибким и имеет доступные инструменты. Python способен все это предоставить, поэтому сегодня мы видим большое количество проектов Python AI.

Python предлагает лаконичный и читаемый код. В то время как за машинным обучением и искусственным интеллектом стоят сложные алгоритмы и универсальные процессы обработки данных, простота Python позволяет разработчикам создавать надежные системы. Они могут приложить все усилия для решения задач машинного обучения, вместо того чтобы сосредотачиваться на технических нюансах языка. Кроме того, Python привлекателен для многих разработчиков, поскольку его легко освоить. Код Python понятен людям, что упрощает создание моделей для машинного обучения. \cite{python_for_ml}

Самым весомым преимуществом языка Python в сфере машинного обучения является наличие open-source библиотек и фреймфорков, содержащих в себе большинство актуальных ML-алгоритмов. Основные задачи и библиотеки для их решения представлены в таблице 2.1:

\begin{table}[ht]
	\caption{Библиотеки Python для машинного обучения}
	\label{table:tech:ml_libs}
	\centering
	\begin{tabular}{|p{0.5\linewidth}|p{0.5\linewidth}|}
		\hline
		\multicolumn{1}{|c|}{Задача машинного обучения} & \multicolumn{1}{|c|}{Библиотеки} \\
		\hline
		Работа с данными и визуализация & pandas, numpy, scipy, matplotlib, seaborn, plotly \\
		\hline
		Классическое машинное обучение & scikit-learn, xgboost, catboost \\
		\hline
		Компьютерное зрение & pytorch, tensorflow, openCV, albumentations \\
		\hline
		Обработка естественного языка (NLP) & pytorch, tensorflow, huggingface, nltk \\
		\hline
	\end{tabular}
\end{table}


\subsection{Jupyter}

Jupyter Notebook --- это open-source веб-приложение, которое можно использовать для создания и обмена документами, содержащими живой код, уравнения, визуализации и текст.

Jupyter-Notebook по умолчанию работает с ядром IPython, так как изначально был частью данного проекта, однако в настоящее время существует более 100 других ядер, которые также можно использовать.

Jupyter-Notebook является одним из основных инструментов аналитиков и специалистов Data Science. Это связано с удобным интерфейсом для аналитических задач. В нем можно писать обычный код, как и в стандартных скриптах языка Python, однако также позволяет писать и запускать bash-скрипты. Также в jupyter есть отдельный набор команд, так называемые «line magics», которые начинаются с символа «\%». К ним относятся как некоторые станжартные bash команды, так испециальные для удобной работы с языком Python, например:

\begin{itemize}
	\item \%conda --- запускает пакетный менеджер Anaconda, позволяет устанавливать различные пакеты и менять среду разработки.
	\item \%debug --- активирует режим отладки. Эта волшебная команда поддерживает два способа активации отладчика. Один из них --- активировать отладчик перед выполнением кода. В этом случае можно установить точку остановки, чтобы выполнить код с точки. Другой --- активировать отладчик в режиме post-mortem. Если только что произошло исключение, данный режим отладки позволяет в интерактивном режиме рассмотреть его traceback. Стоит отметить, что если появится еще одно исключание, информация о предыдущем теряется.
	\item \%env --- позволяет просматривать и менять значения переменных окружения.
	\item \%matplotlib --- настраивает matplotlib для интерактивной работы. Эта функция позволяет активировать интерактивную поддержку matplotlib в любой момент во время сеанса IPython. При этом выполнение данной команды ничего не импортирует в интерактивное пространство имен.
	\item \%timeit --- измеряет время выполнения ячейки.
\end{itemize}


\subsection{Pandas}

Pandas --- это open-source библиотека для языка Python, предоставляющая быстрые, гибкие и выразительные структуры данных, предназначенные для того, чтобы сделать работу с данными простой и интуитивно понятной. Pandas является одним из самых мощных и гибких инструментов анализа и манипулирования данными с открытым исходным кодом. К задачам, решаемым данной библиотекой, относятся:

\begin{itemize}
	\item Простая обработка отсутствующих данных (представленных как NaN, NA или NaT) в виде данных с плавающей запятой, а также данных без плавающей запятой
	\item Изменяемость размера: столбцы можно вставлять и удалять из DataFrame и объектов более высоких размерностей.
	\item Автоматическое и явное выравнивание данных: объекты могут быть явно выровнены по набору меток, или пользователь может просто игнорировать метки и позволить Series, DataFrame и т. д. автоматически выравнивать данные для вас в вычислениях.
	\item Мощная и гибкая группировка по функциям для агрегирования и для преобразования данных.
	\item Упрощенное преобразование необработанных данных из различных структур Python и NumPy в объекты DataFrame.
	\item Интуитивное слияние и объединение наборов данных
	\item Надежные инструменты ввода-вывода для загрузки данных из .csv файлов, баз данных и сохранения/загрузки данных из сверхбыстрого формата HDF5.
	\item Специфические функции для работы с временными рядами: генерация диапазона дат и преобразование частоты, статистика скользящего окна, сдвиг даты и отставание
\end{itemize}

\subsection{Pytorch}

Pytorch --- это open-source библиотека для языка Python, которая предоставляет две основные функции:
\begin{itemize}
	\item Тензорные вычисления (аналогичные тем, что предлагаем библиотека NumPy) с сильным ускорением графического процессора;
	\item Глубокие нейронные сети со встроенной реализацией обратного распространения градиентов,
\end{itemize}

PyTorch предоставляет функционал тензорных вычислений, которые могут проводиться как на обычном процессоре (CPU), так и на графическом процессоре (GPU). К данным операциям относятся индексирование, математические операции, линейная алгебра.

Большинство фреймворков, таких как TensorFlow, Theano, Caffe и CNTK, имеют статическое представление о мире. Нужно построить нейронную сеть и использовать одну и ту же структуру снова и снова. Изменение поведения сети означает, что нужно начинать с нуля, так как для каждой сети создается строгий граф вычислений, в который крайне сложно вносить изменения.
В PyTorch используется метод, называемый обратным автодифференцированием, который позволяет произвольно изменять поведение сети. ~\cite{pytorch_github}

Особого внимания заслуживают отдельные расширения для PyTorch. К ним относятся torchvision, torchtext, torchaudio и другие. Они значительно расширяют общий функционал, добавляют готовые обученные модели, которые можно использовать для всевозможных задач и экспериментов. Относительно новый пакет torch.distributed содержит различные инструменты для оптимизации и распараллеливания обучения и применения нейросетей. Данная библиотека позволяет довольно легко обучать модели на нескольких видеокартах, распараллеливать расчет градиентов, отслеживать использование CPU, GPU, оперативной памяти для дальнейшей оптимизации. Благодаря torch.distributed обучать большие ресурсозатратные модели стало намного проще.

Также в PyTorch входит пакет для деплоя моделей для работы в пользовательских приложениях TorchScript — это способ создания сериализуемых и оптимизируемых моделей из кода PyTorch. TorchScript предоставляет инструменты для поэтапного перехода модели от модуля Python к модулю, который можно запускать на любом другом языке, например, в отдельной программе C++. Это позволяет обучать модели в PyTorch, используя знакомые инструменты в Python, а затем экспортировать модель через TorchScript в производственную среду, где программы Python могут быть невыгодны по причинам производительности и многопоточности. Также пакет содержит ряд инструментов для оптимизации модели, например, квантизация, которая значительно уменьшает размер итогового файла модели, что крайне при использовании на мобильных устройствах.

\subsection{Pytorch lightning}

PyTorch Lightning --- это платформа глубокого обучения для профессиональных исследователей искусственного интеллекта и инженеров по машинному обучению, которым нужна максимальная гибкость без ущерба для производительности в масштабе.

Pytorch Lightning применяет следующую структуру к вашему коду, что делает его повторно используемым и доступным:

\begin{itemize}
	\item Исследовательский код (строится на основе LightningModule).
	\item Инженерный код (полностью покрывается функциональностью класса Trainer).
	\item Несущественный исследовательский код (логирование и т. д., выполняется засчет внешних вызовов).
	\item Данные (можно использовать функционал классов Dataset и DataLoader из pytorch).
\end{itemize}

Используя Lightning, можно «из коробки» тренироваться на нескольких GPU, TPU, CPU, IPU, HPU и даже с 16-битной точностью без изменения кода.

Преимущества перед неструктурированным PyTorch:

\begin{itemize}
	\item Модели становятся аппаратно-независимыми.
	\item Код легко читается, потому что инженерный код абстрагирован.
	\item Легче воспроизвести.
	\item Совершается меньше ошибок, так как основной инженерный код заключен в интерфейсы Lightning.
	\item Сохраняется вся гибкость (LightningModules по-прежнему являются модулями PyTorch), но удаляется масса шаблонов.
	\item Lightning имеет десятки интеграций с популярными инструментами машинного обучения (например, логгирование wandb).
	\item Минимальные накладные расходы на скорость работы (около 300 мс на эпоху по сравнению с чистым PyTorch).
\end{itemize}

Основное преимущество использование Pytorch Lightning --- это встроенный класс Trainer. Он избавляет от необходимости писать под каждую модель полные циклы обучения, которые в большинстве своем похожи но достаточно трудно обобщаемы для разных типов моделей. Для Trainer достаточно обернуть модель в класс LightningModule, реализовать ряд методов, описывающих один шаг обучения и валидации, добавить логирование через встроенный метод log, и этого достаточно, чтобы осуществить полноценное обучение модели одним вызовом trainer, а множество входных аргументов позволяют подстроить процесс под любые нужды.

\subsection{NLTK}

NLTK --- это ведущая платформа для работы с естественным языком в Python. Она предоставляет простые в использовании интерфейсы для более чем 50 массивов текстовых данных и лексических ресурсов, таких как WordNet, а также набор библиотек с инструментами для обработки текста в задачах классификации, токенизации, синтаксического и семантического анализа.

NLTK позволяет решить большинство задач по предобработке текста для различных задач NLP. NLTK содержит инструменты для токенизации текстов, разбиения на слова или предложения, расчета различных статистик на текстовых данных. Также данный пакет содержит реализации большинства статистических языковых моделей, таких как Laplace Smoothing и Kneser-Nay.

\subsection{Huggingface}

Huggingface включает в себя ряд библиотек для машинного обучения, в которых содержатся всевозможные модели для NLP, компьютерного зрения, различные утилиты и большие наборы данных. К Huggingface относятся:

\begin{enumerate}
	\item Transformers --- библиотека для Python, которая предоставляет тысячи предварительно обученных моделей. Эти модели могут применяться на:
	\begin{itemize}
		\item Текстовых данных для таких задач, как классификация текста, извлечение информации, ответы на вопросы, обобщение, перевод, генерация текста на более чем 100 языках.
		\item Изображениях для таких задач, как классификация изображений, обнаружение объектов и сегментация.
		\item Аудио для задач распознавания речи и классификации аудио.
	\end{itemize}

	Модели-трансформеры также могут выполнять задачи в сочетании нескольких типов данных, таких как ответы на вопросы, оптическое распознавание символов, извлечение информации из отсканированных документов, классификация видео и визуальные ответы на вопросы.
	
	Transformers предоставляет API для быстрой загрузки и использования этих предварительно обученных моделей для заданного текста, тонкой настройки их в ваших собственных наборах данных, а затем предоставления доступа к ним сообществу в нашем центре моделей. В то же время каждый модуль Python, определяющий архитектуру, является полностью автономным и может быть изменен для проведения быстрых исследовательских экспериментов.
	
	Transformers поддерживается тремя самыми популярными библиотеками глубокого обучения — Jax, PyTorch и TensorFlow --- с легкой интеграцией между ними. Легко обучить свои модели с помощью одного инструмента, прежде чем загружать их для вывода с помощью другого.
	
	\item Datasets --- библиотека, предоставляющая инструменты для эффективной предобработки данных, а также удобный доступ к большинству наиболее популярных открытых датасетов под различные задачи.
	
	\item Optimum --- инструменты для оптимизации обучения больших моделей-трансформеров.
\end{enumerate}