\lstset{style=fsharpstyle}

\section{Используемые технологии} 
\label{sec:practice:technology_used}

\subsection{Язык Python}

Python — высокоуровневый язык программирования общего назначения с динамической строгой типизацией и автоматическим управлением памятью, ориентированный на повышение производительности разработчика, читаемости кода и его качества, а также на обеспечение переносимости написанных на нём программ. Язык является полностью объектно-ориентированным — всё является объектами. Синтаксис ядра языка минималистичен, за счёт чего на практике редко возникает необходимость обращаться к документации. Сам же язык известен как интерпретируемый и используется в том числе для написания скриптов. Недостатками языка являются зачастую более низкая скорость работы и более высокое потребление памяти написанных на нём программ по сравнению с аналогичным кодом, написанным на компилируемых языках. 
Python стал главным языком мира машинного обучения за счет того, что:

\begin{itemize}
	\item Python обладает большим выбором библиотек и фреймворков.  В научных расчетах используется Numpy, в продвинутых вычислениях — SciPy, в извлечении и анализе данных при помощи методов классического машинного обучения — SciKit-Learn. Для нейросетевых моделей используются такие фреймворки как Tensorflow (больше распространен в production сфере в силу возможностей для реализации распределенных вычислений на нескольких машинах) и Pytorch (распространен в академической среде в силу своей простоты). При этом Python является языком общего назначения, поэтому кроме инструментов для сложных вычислений обладает всеми типичными для других языков инструментами (к примеру, микро-фреймворк для создания веб-сервисов Flask), которые также используются для построения полноценных приложений на основе машинного обучения
	\item Понятность. Python предоставляет разработчику уровень абстракций и конструкций обеспечивающих понятность написанных программ. Это подходит для машинного обучения, потому что сами алгоритмы машинного обучения сложны для понимания. При работе с Python разработчику не нужно уделять много внимания непосредственно написанию кода: все внимание он может сосредоточить на решении более сложных задач, связанных с машинным обучением. Простой синтаксис языка Python помогает разработчику тестировать сложные алгоритмы с минимальной тратой времени на их реализацию.
	\item Общирная поддержка. Еще одно преимущество Python — это обширная поддержка и качественная документация. Существует множество полезных ресурсов о Python, на которых программист может получить помощь и консультацию, находясь на любом этапе разработки.
	\item Следующее преимущество Python в машинном обучении состоит в его гибкости: например, у разработчика есть выбор между объектно-ориентированным подходом и скриптами. Python помогает объединять различные типы данных. Более того, Python особенно удобен для тех разработчиков, которые большую часть кода пишут с помощью IDE
\end{itemize}
Перечисленные выше факторы объясняют, почему Python так активно используется в сфере машинного обучения. Его простота помогает быстро прототипировать сложные нейросетевые модели, проверять и оценивать различные эксперименты с данными.

\subsection{Jupyter}
\subsection{Pandas}
\subsection{Matplotlib}
\subsection{Sentencepiece}
\subsection{Pytorch}
\subsection{ONNX}
