\sectioncentered*{Введение}
\addcontentsline{toc}{section}{Введение}
\label{sec:intro}

В XXI веке самым дорогим ресурсом по праву считается информация. Именно сфера информационных технологий является двигателем прогресса, и в ней сосредоточены самые большие капиталы и наиболее крупные компании, такие как Apple, Google, Amazon и многие другие. Объемы данных, с которыми приходится работать компаниям для максимизации своей эффективности, стремительно растут, что порождает ряд трудностей, связанных с их интерпретацией и использованием. Это делает невозможным их ручную обработку, что приводит к необходимостью автоматизации основных процессов работы с данными. Здесь на помощь приходят технологии машинного обучения.

С развитием информационных технологий все большее внимание уделяется нейронным сетям и искусственному интеллекту. Этому способствует увеличение объемов доступных данных, развитие вычислительной техники, а также новые исследования в сфере машинного обучения. Различные статистические и нейросетевые модели позволяют проводить анализ информации и делают это намного быстрее и, зачастую, точнее человека. Поэтому сфера ИИ стремительно набирает популярность: количество проводимых исследований увеличивается, а спрос на квалифицированные кадры стабильно растет.

Одним из направлений развития ИИ является языковое моделирование. Данная технология находит довольно широкое применение во многих популярных приложениях и сервисах. Основная задача, решаемая языковыми моделями, --- предсказывание следующего элемента некоторой последовательности, чаще всего текстовой. Например, всплывающие подсказки при наборе сообщений в мессенджерах или запросов в брузере, --- это результат применения языковой модели, которая на основе текста, введенного пользователем, выдает наиболее вероятное продолжение. Однако область применения языковых моделей значительно шире. Они играют ключевую роль при распознавании речи, а в настоящее время появляются модели, способные решать задачи по программированию. 

В последние годы на передний план при нейросетевом моделировании выходит работа с неразмеченными, «сырыми» данными. Большую часть таких данных составляют аудио и текст. Это привело к развитию алгоритмов unsupervised learning, к которым относятся и вариации языковых моделей. Стоит также отметить развитие подхода domain adaptation, или «адаптация домена», который позволяет применять такие модели для задач с минимальным количеством размеченных данных и получать отличное качество, что стало прорывом в сфере ИИ.

Относительно недавно начали проводиться исследования по созданию языковых моделей для менее распространенных языков. Наилучшие языковые модели были получены на таких доменах, как английский, русский и французский, однако в мире существуют сотни менее популярных языков, для которых данных существенно меньше, что усложняет создание хорошей модели. На данный момент одним из основных направлений в языковом моделировании является создание качественных моделей при наличии минимального объема данных.

Несмотря на всю привлекательность использования огромных нейросетей для решения различных задач, у них есть довольно большой недостаток --- вычислительная стоимость. Лучшая на данный момент языковая модель, GPT-3, была получена при непрерывном использовании многих десятков графических ускорителей в течение нескольких месяцев. Стоимость техники, необходимой для таких вычислений, достаточно высокая, что делает создание тяжелых моделей дорогостоящим. Поэтому в развитии нейросетей важную роль играет применение различных оптимизаций и подходов, ускоряющих обучение и снижающих количество параметров без потерь в конечном качестве модели, а также всевозможные механизмы параллельного обучения.

В данной работе будут проанализированы различные алгоритомы языкового моделирования, проведен ряд экспериментов по обучению моделей на домене белорусского языка. В результате будет выполнен сравнительный анализ полученных моделей по ряду метрик и сделаны выводы на основе полученных результатов.
