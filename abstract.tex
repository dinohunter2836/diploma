\sectioncentered*{Реферат}
\thispagestyle{empty}
%%
%% ВНИМАНИЕ: этот реферат не соответствует СТП-01 2013
%% пример оформления реферата смотрите здесь: http://www.bsuir.by/m/12_100229_1_91132.docx 
%%

ЯЗЫКОВАЯ МОДЕЛЬ ДЛЯ БЕЛОРУССКОГО ЯЗЫКА НА ОСНОВЕ НЕЙРОННЫХ СЕТЕЙ: дипломная работа / А.Б.Яковлев. –  Минск: БГУИР, 2022, – п.з – \pageref*{LastPage} с., чертежей (пла-катов) – 6 л.

\emph{Ключевые слова}: языковые модели; нейронные сети; машинное обучение; генерация текста; рекуррентные сети.

%Дипломный проект выполнен на 6 листах формата А1 с пояснительной запиской на~\pageref*{LastPage} страницах, без приложений справочного или информационного характера. 
%Пояснительная записка включает \total{section}~глав, \totfig{}~рисунков, \tottab{}~таблиц, \toteq{}~формул и \totref{}~литературных источников.

Объектом исследований являются нейросетевые языковые модели.

Целью дипломного проекта является разработка нейросетевой языковой модели, которую можно использовать для генерации текстовых последовательностей на белорусском языке.

При разработке приложения использовался следующий набор технологий: Python, Jupyter, PyTorch, Pytorch Lightning, Pandas, Wandb, NLTK, HuggingFace.

В первом разделе данной работы осуществляется анализ сферы языкового моделирования, рассматриваются всевозможные архитектуры и модели.

Второй раздел содержит информацию о технологиях, используемых для создания языковой модели.

В третьем разделе проведен процесс создания всех необходимых скриптов для сбора и обработки данных, обучения моделей и логирования результатов.

Четвертый раздел содержит описание проведенных экспериментов.

В пятом разделе приведено технико-экономическое обоснование эффек-тивности разработки программного средства.

В разделе заключение содержатся выводы по данному дипломному проекту.

Дипломный проект является завершенным, поставленная задача решена в полной мере, присутствует возможность дальнейшего проведения исследований и улучшения качества моделей.

\clearpage
