{
  \newgeometry{top=1.25cm,bottom=1.25cm,right=1cm,left=2cm,twoside}
  \thispagestyle{empty}
  \setlength{\parindent}{0em}

  \newcommand{\lineunderscore}{\uline{\hspace*{\fill}}}
  
  \newcommand{\raisedrule}[1]{\rule[#1]{\linegoal}{0.45pt}}
  \newcommand{\longunderline}[1]{\ul{#1}\raisedrule{-.75ex}}

  
  \begin{center}
    Министерство образования Республики Беларусь\\
    Учреждение образования\\
    БЕЛОРУССКИЙ ГОСУДАРСТВЕННЫЙ УНИВЕРСИТЕТ \\
    ИНФОРМАТИКИ И РАДИОЭЛЕКТРОНИКИ\\[1em]
  

  \begin{minipage}{\textwidth}
    \begin{flushleft}
      \begin{tabular}{ p{0.20\textwidth}p{0.31\textwidth}p{0.20\textwidth}p{0.20\textwidth} @{} }
        Факультет & КС и С & Кафедра & Информатики \\
        Специальность   & 1-40 04 01 & Специализация & 
      \end{tabular}
    \end{flushleft}
  \end{minipage}\\[1em]

  \begin{minipage}{\textwidth}
    \begin{flushright}
      \begin{tabular}{p{0.40\textwidth}}
        УТВЕРЖДАЮ \\[0.5em]
        \underline{\hspace*{7em}} Зав. кафедрой \\
        <<\underline{\hspace*{4ex}}>> \underline{\hspace*{7em}} 2022 г.
      \end{tabular}
    \end{flushright}
  \end{minipage}\\[1em]

  ЗАДАНИЕ\\
  по дипломному работе студента\\

  \uline{\hspace*{\fill}\textbf{Яковлева Артура Бахтияровича}\hspace*{\fill}}\\
  {\small (фамилия, имя, отчество) }
	
  \end{center}

  \uline{\mbox{\hspace*{0.2em}}1. Тема работы: \quad Языковая модель для белорусского языка на основе нейронных сетей.\hfill\mbox{\hspace*{15.5em}}} \\
  
  утверждена приказом по университету от \hspace*{1em} <<\uline{\mbox{\hspace*{0.5em}}13\mbox{\hspace*{0.5em}}}>> \uline{\mbox{\hspace*{1em}}Апреля\mbox{\hspace*{1em}}} 2022 г.  \No \uline{\mbox{\hspace*{0.5em}}961-c\mbox{\hspace*{1em}}}

  \vspace{1em}

  \hspace*{0.2em}2. Срок сдачи студентом законченной работы: \lineunderscore

  \vspace{1em}

  \hspace*{0.2em}3. Исходные данные к работе \hspace*{1em}
  \uline{Тип операционной системы -- ОС Linux;\hfill}\\
  \uline{\mbox{\hspace*{0.2em}} Язык программирования -- Python; Библиотеки -- Pandas, Jupyter, Pytorch,\hfill}\\ \uline{\mbox{\hspace*{0.2em}} Pytorch Lightning, NLTK, Sentencepiece, Huggingface.\hfill}\\
  \uline{\mbox{\hspace*{0.2em}} Цель работы: исследование алгоритмов и архитектур нейросетевых моделей для решения задачи языкового моделирования\hfill} \\
  %\lineunderscore\\

  \vspace{1em}

  \hspace*{0.2em}4. Содержание пояснительной записки (перечень подлежащих разработке вопросов) \\
  \uline{\mbox{\hspace*{0.5em}} Введение\hspace*{\fill}}\\
  \uline{\mbox{\hspace*{0.5em}} 1 Обзор предметной области\hspace*{\fill}}\\
  \uline{\mbox{\hspace*{0.5em}} 2 Обзор используемых технологий\hspace*{\fill}}\\
  \uline{\mbox{\hspace*{0.5em}} 3 Реализация языковых моделей\hspace*{\fill}}\\
  \uline{\mbox{\hspace*{0.5em}} 4 Анализ языковых моделей\hspace*{\fill}}\\
  \uline{\mbox{\hspace*{0.5em}} 5 \mbox{Технико\hyphэкономическое} обоснование\hspace*{\fill}}\\
  \uline{\mbox{\hspace*{0.5em}} Заключение\hspace*{\fill}}\\
  \uline{\mbox{\hspace*{0.5em}} Список использованных источников\hspace*{\fill}}\\
  \uline{\mbox{\hspace*{0.5em}} Приложение А - Текст программы\hspace*{\fill}}\\
  \lineunderscore\\
  \lineunderscore

  \vspace{1em}
  
  \thispagestyle{empty}
  
  \hspace*{0.2em}5. Перечень графического материала (с точным указанием обязательных чертежей):
  \lineunderscore\\
  \lineunderscore\\
  \lineunderscore\\
  \lineunderscore\\
  \lineunderscore\\
  \lineunderscore\\
  \lineunderscore\\
  \lineunderscore

  \vspace{1em}

  \uline{\hspace*{0.2em}6. Технико-экономическое обоснование разработки программного средства \hspace*{\fill}\\\hspace*{0.2em} «Языковая модель для белорусского языка на основе нейронных сетей»\hspace*{\fill}}\\
  \uline{\hspace*{0.2em}Характеристика программного средства, разрабатываемого для реализации на \hspace*{\fill}\\\hspace*{0.2em} рынке\hspace*{\fill}}\\
  \uline{\hspace*{0.2em}Расчет инвестиций в разработку программного средства для его реализации на \hspace*{\fill}\\\hspace*{0.2em} рынке\hspace*{\fill}}\\
  \uline{\hspace*{0.2em}Расчет экономического эффекта от реализации программного средства на рынке\hspace*{\fill}} \\
  \uline{\hspace*{0.2em}Расчет показателей экономической эффективности разработки и реализации \hspace*{\fill}\\\hspace*{0.2em}программного средства на рынке\hspace*{\fill}}\\
  \hspace*{0.2em}Задание выдал: \hfill{} \uline{\hspace*{6em}} / С.\,В.~Наркевич /   

  \vspace{1em}


  \begin{center}
    \textbf{КАЛЕНДАРНЫЙ ПЛАН}
  \end{center}

  \begin{tabular}{| m{0.48\textwidth} 
                  | >{\centering}m{0.08\textwidth}
                  | >{\centering}m{0.19\textwidth}  
                  | >{\centering\arraybackslash}m{0.16\textwidth}|}
    \hline \centering{Наименование этапов дипломного проекта (работы)} & Объем этапа, \% & Срок выполнения этапов & Примечание \\
    \hline Обзор предметной области, & & & \\
    \hline разработка технического задания & 20 & 01.02--14.02 & \\
    \hline Подготовка моделей & & & \\
    \hline машинного обучения & 15 & 15.02--10.03 & \\
    \hline Дизайн экспериментов для анализа & & & \\
    \hline моделей машинного обучения & 15 & 11.03--27.03 & \\
    \hline Программная реализация  & & & \\
    \hline моделей и экспериментов & 30 & 28.03--08.05 & \\
    \hline Оформление пояснительной  & & & \\
    \hline записки и графического материала & 20 & 09.05--31.05 & \\
    \hline
  \end{tabular}

  \vspace{2em}

  Дата выдачи задания: \uline{\hspace*{6em}} \hspace{2ex} Руководитель \hfill{} \uline{\hspace*{4em}} /~М.\,А.~Калугина~/

  \vspace{1em}

  Задание принял к исполнению \hfill{} \uline{\hspace*{4em}} /~А.\,Б.~Яковлев~/

  \restoregeometry
}
