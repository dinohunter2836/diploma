\newcommand{\byr}{Br}

\section{Технико-экономическое обоснование разработки программного модуля «Языковая модель для белорусского языка на основе нейронных сетей»}

% Begin Calculations

\FPeval{\pmSalaryPerMonth}{3200}
\FPeval{\pmSalaryPerHourExact}{\pmSalaryPerMonth / 168}
\FPround{\pmSalaryPerHour}{\pmSalaryPerHourExact}{1}
\FPeval{\pmWorkingHours}{ 240 }
\FPeval{\pmTotal}{clip( \pmSalaryPerHour * \pmWorkingHours )}

\FPeval{\devSalaryPerMonth}{2600}
\FPeval{\devSalaryPerHourExact}{\devSalaryPerMonth / 168}
\FPround{\devSalaryPerHour}{\devSalaryPerHourExact}{1}
\FPeval{\devWorkingHours}{ 480 }
\FPeval{\devTotal}{clip( \devSalaryPerHour * \devWorkingHours )}

\FPeval{\mlSalaryPerMonth}{2740}
\FPeval{\mlSalaryPerHourExact}{\mlSalaryPerMonth / 168}
\FPround{\mlSalaryPerHour}{\mlSalaryPerHourExact}{1}
\FPeval{\mlWorkingHours}{ 480 }
\FPeval{\mlTotal}{clip( \mlSalaryPerHour * \mlWorkingHours )}

\FPeval{\teamSalarySum}{clip (\pmTotal + \devTotal + \mlTotal)}
\FPeval{\teamSalaryAdditional}{clip (\teamSalarySum * 0.5)}
\FPeval{\teamSalaryTotal}{clip (\teamSalarySum + \teamSalaryAdditional)}

\FPeval{\extraSalaryCoeff}{13}
\FPeval{\extraSalary}{clip (\teamSalaryTotal * 13 / 100)}

\FPeval{\socialSpendingsExact}{clip ((\teamSalaryTotal + \teamSalaryAdditional) * 34.6 / 100)}
\FPround{\socialSpendings}{\socialSpendingsExact}{2}

\FPeval{\saleCoeff}{4}
\FPeval{\saleCost}{ clip (\teamSalaryTotal * \saleCoeff / 100) }

\FPeval{\otherSpendingsCoeff}{35}
\FPeval{\otherSpendingsExact}{clip(\teamSalaryTotal * \otherSpendingsCoeff / 100)}
\FPround{\otherSpendings}{\otherSpendingsExact}{2}

\FPeval{\totalCost}{clip(\teamSalaryTotal + \extraSalary + \socialSpendings + \saleCost + \otherSpendings)}

\FPeval{\licenseCost}{1200}
\FPeval{\copies}{200}
\FPeval{\totalIncome}{clip (\licenseCost * \copies * 30 / 100)}
\FPeval{\totalGain}{ clip(\totalIncome - \totalCost) }

\FPeval{profitabilityExact}{clip (100 * (\totalIncome - \totalCost) / \totalCost)}
\FPround{\profitability}{\profitabilityExact}{1}

% End Calculations

\subsection{Характеристика программного средства, разрабатываемого для реализации на рынке}
Языковая модель позволяет решать различные задачи, связанные с обработкой и генерацией текстов. К ним можно отнести:
\begin{itemize}
	\item[•] генерация текстовых последовательностей;
	\item[•] оценка текстовых последовательностей;
	\item[•] создание векторных представлений текстовых документов.
\end{itemize}

Значительное количество современных сервисов, так или иначе работающие с большими объемами текстовых данных, используют языковые модели. Например, в наиболее популярных социальных сетях или браузерах при вводе сообщения или запроса появляются подсказки, которые часто совпадают с тем, что хотел ввести пользователь. Данная функция как раз и основывается на применении языковых моделей, которые по предоставленному контексту могут достаточно точно предсказывать следующие слова.
Также относительно недавно появился сервис Яндекса Зелибоба, который по нескольким предложениям может создать осмысленный рассказ. В его основе лежит языковая модель, содержащая более двух миллиардов параметров. Однако еще ни в одном приложении нет подобного функционала для белорусского языка, так как еще никто не создал соответствующую нейросеть. Целью данной работы и является это исправить, предоставив рынку такую технологию, что позволит расширить функционал множеству приложений, и автоматизировать ряд процессов по работе с текстовыми данными.

Кроме описанных выше примеров, к области применения можно отнести голосовые помощники. Языковые модели являются составной частью технологии распознавания речи, и приобретение на рынке может быть более выгодной альтернативой использованию внутренних ресурсов для обучения собственных моделей.

Данный программный продукт разрабатывается ООО «Эквай». К предполагаемым покупателям можно отнести:
\begin{itemize}
	\item[•] государственные структуры (органы власти, СМИ, архивы и др.), так как они сталкиваются с большим потоком документов, в том числе на белорусском языке;
	\item[•] соцсети, интернет-магазины и другие сервисы, ориентированные на белорусский рынок;
	\item[•] компании, занимающиеся разработкой голосовых помощников, таких как Алиса от Яндекса или Alexa от Amazon;
	\item[•] стартапы, связанные с обработкой текстовых данных.
\end{itemize}

\subsection{Расчет инвестиций в разработку программного средства для его реализации на рынке}

Цена программного продукта будет определяться на основе инвестиций в создание данного продукта компанией-разработчиком.

\subsubsection{Затраты на основную заработную плату команды разработчиков}. Команда разработчиков состоит из руководителя разработки, программиста и специалиста по машинному обучению.
Данные по заработной плате команды разработчиков предоставлены ООО «Эквай» на 14.04.2022.
Затраты на основную заработную плату команды разработчиков $\text{З}_0$ рассчитываются по формуле 6.1:
\begin{equation}
	\text{3}_o = K_{\text{пр}} \cdot \sum_{i=1}^{n}{\text{З}_{\text{чi}} \cdot t_i}
\end{equation}
\begin{explanation}
	где $K_{\text{пр}}$ – коэффициент премий и иных стимулирующих выплат, 1,5; \\
	$\text{З}_{\text{чi}}$ – часовой оклад исполнителя i-й категории; \\
	$n$ – категории исполнителей; \\
	$t_i$ – трудоемкость работ исполнителя i-й категории.
\end{explanation}

Результаты расчета затрат на основную заработную плату команды разработчиков представлены в таблице 6.1.

\begin{table}[ht]
\caption{Расчет затрат на основную заработную плату команды разработчиков}
\label{table:econ:function_sizes}
\centering
  \begin{tabular}{|>{\centering}m{0.2\textwidth}|>{\centering}m{0.2\textwidth}|>{\centering}m{0.2\textwidth}|>{\centering}m{0.2\textwidth}|c|}
		\hline
		Категория исполнителя & Месячный оклад, р. & Часовой оклад, р. & Трудоемкость работ, ч. & Итого, р. \\
		\hline
		Руководитель разработки & \num{\pmSalaryPerMonth} & \num{\pmSalaryPerHour} & \num{\pmWorkingHours} & \num{\pmTotal} \\
		\hline
		Программист & \num{\devSalaryPerMonth} & \num{\devSalaryPerHour} & \num{\devWorkingHours} & \num{\devTotal} \\
		\hline
		Специалист по машинному обучению & \num{\mlSalaryPerMonth} & \num{\mlSalaryPerHour} & \num{\mlWorkingHours} & \num{\mlTotal} \\
		\hline
		\multicolumn{4}{|l|}{Итого} & \num{\teamSalarySum} \\
		\hline
		\multicolumn{4}{|l|}{Премии и иные стимулирующие выплаты (50\%)} & \num{\teamSalaryAdditional} \\
		\hline
		\multicolumn{4}{|l|}{Всего затрат на основную плату разработчиков} & \num{\teamSalaryTotal} \\
		\hline
		\end{tabular}
\end{table}

\subsubsection{Затраты на дополнительную заработную плату команды разработчиков}. Затраты на дополнительную заработную плату команды разработчиков (Зд) рассчитываются по формуле 6.2:

\begin{equation}
	\text{З}_\text{д} = \frac{\text{З}_o \cdot \text{Н}_\text{д}}{100},
\end{equation}
\begin{explanation}
	где $H_{\text{д}}$ -- норматив норматив дополнительной заработной платы, 13\%.
\end{explanation}

Размер затрат на дополнительную заработную плату команды разработчиков составит:

$$
	\text{З}_\text{д} = \frac{\text{З}_o \cdot \text{Н}_\text{д}}{100} = \frac{\num{\teamSalaryTotal} \cdot \num{\extraSalaryCoeff}}{100} = \num{\extraSalary} \text{р.}
$$

\subsubsection{Отчисления на социальные нужды}. Отчисления на социальные нужды $P_{\text{соц}}$ рассчитываются по формуле 6.3:

\begin{equation}
	P_{\text{соц}} = \frac{(\text{З}_o + \text{З}_{\text{д}}) \cdot H_{\text{соц}}}{100},
\end{equation}
\begin{explanation}
	где $H_{\text{соц}}$ -- норматив отчислений в фонд социальной защиты населения \\Белгосстрах, 34,6\%.
\end{explanation}

$$
P_{\text{соц}} = \frac{\text{З}_o + \text{З}_{\text{д}} \cdot H_{\text{соц}}}{100} = \frac{(\num{\teamSalaryTotal} + \num{\teamSalaryAdditional}) \cdot 34.6}{100} = \num{\socialSpendings} \text{р}
$$

\subsubsection{Расходы на реализацию}. Расходы на реализацию рассчитываются по формуле 6.4:

\begin{equation}
P_p = \frac{\text{З}_{\text{o}} \cdot H_p}{100}
\end{equation}
\begin{explanation}
	где $H_o$ -- норматив расходов на реализацию, 4\%.
\end{explanation}

$$
P_p = \frac{\text{З}_{\text{o}} \cdot H_p}{100} = \frac{\num{\saleCoeff} \cdot \num{\teamSalaryTotal}}{100} = \num{\saleCost} \text{p}
$$

\subsubsection{Прочие расходы}. Прочие расходы ($P_{\text{пр}}$) рассчитываются по формуле 6.5:

\begin{equation}
	P_{\text{соц}} = \frac{\text{З}_o \cdot H_{\text{пр}}}{100},
\end{equation}
\begin{explanation}
	где $H_{\text{пр}}$ -- норматив прочих расходов, $\num{\otherSpendingsCoeff}$\%.
\end{explanation}

Размер прочих расходов составит:

$$
P_{\text{соц}} = \frac{\text{З}_o \cdot H_{\text{пр}}}{100} = \frac{\num{\teamSalaryTotal} \cdot \num{\otherSpendingsCoeff}}{100} = \num{\otherSpendings} \text{р}
$$

\subsubsection{Общая сумма затрат на разработку программного продукта}. Общая сумма затрат на разработку программного продукта рассчитывается по формуле 6.6:

\begin{equation}
	\text{З}_\text{р} = \text{З}_o + \text{З}_\text{д} + P_\text{соц} + P_\text{пр}
\end{equation}

Результаты расчета общей суммы затрат на разработку представлены в таблице 6.2.

\begin{table}[ht]
	\caption{Затраты на разработку программного продукта}
	\label{table:econ:total_cost}
	\centering
	\begin{tabular}{|>{\raggedright}m{0.7\textwidth}|c|}
		\hline
		\multicolumn{1}{|c|}{Статья затрат} & Сумма, р. \\
		\hline
	    Основная заработная плата команды разработчиков & \num{\teamSalaryTotal} \\
		\hline
		Дополнительная заработная плата команды разработчиков & \num{\extraSalary} \\
		\hline
		Отчисления на социальные нужды & \num{\socialSpendings} \\
		\hline
		Расходы на реализацию & \num{\saleCost}
		\\
		\hline
		Прочие затраты & \num{\otherSpendings} \\
		\hline
		Общая сумма затрат на разработку ($\text{З}_{\text{р}}$) & \num{\totalCost} \\
		\hline		
	\end{tabular}
\end{table}

\subsection{Расчет экономического эффекта от реализации программного средства на рынке}

Рассчитаем экономический эффект от разработки программного модуля для организации-разработчика. Для программного продукта, который предполагается реализовывать на рынке, экономическим эффектом будет прирост чистой прибыли от продажи потребителям.
Оценим количество потребителей, опираясь на целевую аудиторию:
\begin{itemize}
	\item[•] до 100 государственных учреждений;
	\item[•] до 200 различный социальных сетей и интернет-магазинов;
	\item[•] до 20 компаний-разработчиков голосовых помощников;
	\item[•] до 50 стартапов, так или иначе занимающихся обработкой текстовых данных.
\end{itemize}

На основе проведенного анализа рынка оценим предполагаемое количество реализованных лицензий за год в $N=\num{\copies}$ копий.

Для оценки рыночной стоимости данного продукта, сравним его с существующими аналогами. Google Cloud предоставляет API для решения различных задач по анализу текстовых документов ~\cite{cloud_pricing}. Тариф для обработки 20 тысяч запросов в месяц составляет порядка 40\$ или 480\$ в год. По курсу Национального Банка на 19.04.2022, составляющему 2.8 BYN за 1\$, получаем годовую стоимость такого продукта 1344 BYN. Стоит отметить, что 20 000 --- это оценка нижней границы на количество запросов, для большинства сервисов оно будет значительно больше, тем самым увеличивая общие затраты.

На основе анализа рынка можем считать конкурентной отпускную цену $\text{Ц}_{\text{отп}} = \num{\licenseCost}$ р. за годовую лицензию.
Поскольку организация-разработчик ООО «Эквай» является резидентом Парка высоких технологий, то она освобождена от уплаты налога на добавленную стоимость и налога на прибыль ($H_{\text{п}} = 0$), и прирост чистой прибыли можно рассчитать по формуле 6.7:

\begin{equation}
	\Delta{\text{П}_{\text{ч}}^{\text{р}}} = \frac{\text{Ц}_{\text{отп}} \cdot N \cdot P_{\text{пр}}}{100}
\end{equation}
\begin{explanation}
	где $\text{Ц}_{\text{отп}}$ -- отпускная цена копии (лицензии) программного средства, р.,  \\
	$N$ -- количество копий (лицензий) программного средства, \\
	$P_{\text{пр}}$ -- рентабельность продаж копий (лицензий), составляет 30\%.
\end{explanation}

Прирост чистой прибыли составит:

$$
\Delta{\text{П}_{\text{ч}}^{\text{р}}} = \frac{\text{Ц}_{\text{отп}} \cdot N \cdot P_{\text{пр}}}{100} = \frac{\num{\licenseCost} \cdot \num{\copies} \cdot 30}{100} = \num{\totalIncome} \text{р.}
$$

\subsection{Расчет показателей экономической эффективности разработки и реализации программного средства на рынке}

Так как экономический эффект за год превышает затраты на разработку, экономическая оценка эффективности производится с помощью расчета рентабельности инвестиций по формуле 6.8:

\begin{equation}
	ROI = \frac{\Delta{\text{П}_{\text{ч}}^{\text{р}}} - \text{З}_{\text{р}}}{\text{З}_{\text{р}}} \cdot 100\%,
\end{equation}
\begin{explanation}
	где $\Delta{\text{П}_{\text{ч}}^{\text{р}}}$ -- прирост чистой прибыли, р., \\
	$\text{З}_{\text{р}}$ -- затраты на его разработку и реализацию.
\end{explanation}

Рентабельность инвестиций составит:

$$
ROI = \frac{\Delta{\text{П}_{\text{ч}}^{\text{р}}} - \text{З}_{\text{р}}}{\text{З}_{\text{р}}} \cdot 100\% = \frac{\num{\totalIncome} - \num{\totalCost}}{\num{\totalCost}} \cdot 100\% = \num{\profitability}\%
$$

Экономические показатели эффективности разработки программного продукта представлены в таблице 6.3.

\begin{table}[ht]
	\caption{Экономические показатели эффективности разработки программного продукта}
	\label{table:econ:gain}
	\centering
	\begin{tabular}{|>{\raggedright}m{0.7\textwidth}|c|}
		\hline
		\multicolumn{1}{|c|}{Экономический показатель} & Значение \\
		\hline
		Затраты на разработку & \num{\totalCost}p. \\
		\hline
		Прирост чистой прибыли & \num{\totalIncome}p. \\
		\hline
		Рентабельность ($ROI$) & \num{\profitability}\%\\
		\hline		
	\end{tabular}
\end{table}


Инвестиции в разработку можно считать целесообразными, так как их рентабельность превышает ставку Национального Банка по депозитам, составляющую 11\% на 19.04.2022. 










